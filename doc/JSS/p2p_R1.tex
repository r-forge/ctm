
\documentclass[12pt]{article}

\let\Prob\P
\usepackage{amsfonts,amstext,amsmath,amssymb,amsthm}

\usepackage{hyperref}

\usepackage[utf8]{inputenc}

\usepackage{natbib}

% File with math commands etc.

%%% mlt
%% rv
\newcommand{\rZ}{Z}
\newcommand{\rY}{Y}
\newcommand{\rX}{\mX}
\newcommand{\rz}{z}
\newcommand{\ry}{y}
\newcommand{\rx}{\xvec}
\newcommand{\erx}{x}
%% sigma algebra
\newcommand{\sA}{\mathfrak{A}}
\newcommand{\sAZ}{\mathfrak{B}}
\newcommand{\sAY}{\mathfrak{C}}
\newcommand{\esA}{A}
\newcommand{\esAZ}{B}
\newcommand{\esAY}{C}
%% sample spaces
\newcommand{\sam}{\Omega}
\newcommand{\samZ}{\RR}
\newcommand{\samY}{\Xi}
\newcommand{\samX}{\chi}
%% measureable spaces
\newcommand{\ms}{(\sam, \sA)}
\newcommand{\msZ}{(\samZ, \sAZ)}
\newcommand{\msY}{(\samY, \sAY)}
%% probability spaces
\newcommand{\ps}{(\sam, \sA, \Prob)}
\newcommand{\psZ}{(\samZ, \sAZ, \Prob_\rZ)}
\newcommand{\psY}{(\samY, \sAY, \Prob_\rY)}
%% distributions
\newcommand{\pZ}{F_\rZ}
\newcommand{\pY}{F_\rY}
\newcommand{\hatpY}{\hat{F}_{\rY,N}}
\newcommand{\hatpYx}{\hat{F}_{\rY | \rX = \rx, N}}
\newcommand{\pN}{\Phi}
\newcommand{\pSL}{F_{\SL}}
\newcommand{\pMEV}{F_{\MEV}}
\newcommand{\pSW}{F_{\SW}}
\newcommand{\pYx}{F_{\rY | \rX = \rx}}
\newcommand{\pYA}{F_{\rY | \rX = A}}
\newcommand{\pYB}{F_{\rY | \rX = B}}
\newcommand{\qZ}{F^{-1}_\rZ}
\newcommand{\qY}{F^{-1}_\rY}
\newcommand{\dZ}{f_\rZ}
\newcommand{\dY}{f_\rY}
\newcommand{\hatdY}{\hat{f}_\rY}
\newcommand{\dYx}{f_{\rY | \rX = \rx}}
\newcommand{\hazY}{\lambda_\rY}
\newcommand{\HazY}{\Lambda_\rY}
\newcommand{\hathazY}{\hat{\lambda}_\rY}
\newcommand{\hatHazY}{\hat{\Lambda}_\rY}
%% measures
\newcommand{\measureY}{\mu}
\newcommand{\lebesgue}{\mu_L}
\newcommand{\counting}{\mu_C}
%% trafo
\newcommand{\g}{g}
\newcommand{\h}{h}
\newcommand{\s}{\svec}
\newcommand{\hY}{h_\rY}
\newcommand{\hx}{h_\rx}
\newcommand{\hs}{\mathcal{H}}
\newcommand{\basisy}{\avec}
\newcommand{\bern}[1]{\avec_{\text{Bs},#1}}
\newcommand{\bernx}[1]{\bvec_{\text{Bs},#1}}
\newcommand{\basisx}{\bvec}
\newcommand{\basisyx}{\cvec}
\newcommand{\m}{m}
\newcommand{\lik}{\mathcal{L}}
\newcommand{\parm}{\varthetavec}
\newcommand{\eparm}{\vartheta}
\newcommand{\dimparm}{P}
\newcommand{\dimparmx}{Q}
\newcommand{\shiftparm}{\betavec}
\newcommand{\eshiftparm}{\beta}

\newcommand{\ie}{\textit{i.e.}~}
\newcommand{\eg}{\textit{e.g.}~}

\renewcommand{\Prob}{\mathbb{P}}
\newcommand{\Ex}{\mathbb{E}}
\newcommand{\RR}{\mathbb{R}}
\newcommand{\eps}{\varepsilon}
\newcommand{\prodname}{tensor }
\newcommand{\Null}{\mathbf{0}}
\newcommand{\FI}{\mF}

\usepackage{dsfont}
\newcommand{\I}{\mathds{1}}



\def \dsP {\text{$\mathds{P}$}}
\def \dsE {\text{$\mathds{E}$}}
\def \dsR {\text{$\mathds{R}$}}
\def \dsN {\text{$\mathds{N}$}}


% Math Operators

 \DeclareMathOperator{\logit}{logit}
 \DeclareMathOperator{\LRT}{LRT}
 \DeclareMathOperator{\RLRT}{RLRT}
 \DeclareMathOperator{\Cov}{Cov}
 \DeclareMathOperator{\Cor}{Cor}
 \DeclareMathOperator{\Var}{Var}
 \DeclareMathOperator{\EW}{\dsE}
 \DeclareMathOperator{\D}{D}
 \DeclareMathOperator{\Bias}{Bias}
 \DeclareMathOperator{\MSE}{MSE}
 \DeclareMathOperator{\PLS}{PLS}
 \DeclareMathOperator{\rank}{rank}
 \DeclareMathOperator{\ncol}{ncol}
 \DeclareMathOperator{\pen}{pen}
 \DeclareMathOperator{\const}{const}
 \DeclareMathOperator{\diag}{diag}
 \DeclareMathOperator{\blockdiag}{blockdiag}
 \DeclareMathOperator{\df}{df}
 \DeclareMathOperator{\trace}{tr}
 \DeclareMathOperator{\iid}{i.i.d.}
 \DeclareMathOperator{\ind}{ind.}
 \DeclareMathOperator{\obs}{obs}
 \DeclareMathOperator{\acos}{acos}
 \DeclareMathOperator{\spat}{spat}
 \DeclareMathOperator{\fix}{{fix}}
 \DeclareMathOperator{\ran}{{ran}}
 \DeclareMathOperator*{\argmin}{{arg\,min}}
 \DeclareMathOperator*{\argmax}{{arg\,max}}
 \DeclareMathOperator{\BIC}{{BIC}}
 \DeclareMathOperator{\DIC}{{DIC}}
 \DeclareMathOperator{\AIC}{{AIC}}
 \DeclareMathOperator{\mAIC}{{mAIC}}
 \DeclareMathOperator{\cAIC}{{cAIC}}

% Distributions

 \DeclareMathOperator{\ND}{N}
 \DeclareMathOperator{\TND}{TN}
 \DeclareMathOperator{\UD}{U}
 \DeclareMathOperator{\GaD}{Ga}
 \DeclareMathOperator{\tD}{t}
 \DeclareMathOperator{\IGD}{IG}
 \DeclareMathOperator{\IWD}{IW}
 \DeclareMathOperator{\PoD}{Po}
 \DeclareMathOperator{\ExpD}{Exp}
 \DeclareMathOperator{\LapD}{Lap}
 \DeclareMathOperator{\MuD}{Mu}
 \DeclareMathOperator{\DirD}{Dir}
 \DeclareMathOperator{\PDD}{PD}
 \DeclareMathOperator{\BeD}{Be}
 \DeclareMathOperator{\BD}{B}
 \DeclareMathOperator{\DPD}{DP}
 \DeclareMathOperator{\KSD}{KS}
 \DeclareMathOperator{\SL}{SL}
 \DeclareMathOperator{\MEV}{MEV}
 \DeclareMathOperator{\SW}{SW}
 \DeclareMathOperator{\Chi1}{\chi^2_1}



% Boldface vectors and matrices

\def \avec {\text{\boldmath$a$}}    \def \mA {\text{\boldmath$A$}}
\def \bvec {\text{\boldmath$b$}}    \def \mB {\text{\boldmath$B$}}
\def \cvec {\text{\boldmath$c$}}    \def \mC {\text{\boldmath$C$}}
\def \dvec {\text{\boldmath$d$}}    \def \mD {\text{\boldmath$D$}}
\def \evec {\text{\boldmath$e$}}    \def \mE {\text{\boldmath$E$}}
\def \fvec {\text{\boldmath$f$}}    \def \mF {\text{\boldmath$F$}}
\def \gvec {\text{\boldmath$g$}}    \def \mG {\text{\boldmath$G$}}
\def \hvec {\text{\boldmath$h$}}    \def \mH {\text{\boldmath$H$}}
\def \ivec {\text{\boldmath$i$}}    \def \mI {\text{\boldmath$I$}}
\def \jvec {\text{\boldmath$j$}}    \def \mJ {\text{\boldmath$J$}}
\def \kvec {\text{\boldmath$k$}}    \def \mK {\text{\boldmath$K$}}
\def \lvec {\text{\boldmath$l$}}    \def \mL {\text{\boldmath$L$}}
\def \mvec {\text{\boldmath$m$}}    \def \mM {\text{\boldmath$M$}}
\def \nvec {\text{\boldmath$n$}}    \def \mN {\text{\boldmath$N$}}
\def \ovec {\text{\boldmath$o$}}    \def \mO {\text{\boldmath$O$}}
\def \pvec {\text{\boldmath$p$}}    \def \mP {\text{\boldmath$P$}}
\def \qvec {\text{\boldmath$q$}}    \def \mQ {\text{\boldmath$Q$}}
\def \rvec {\text{\boldmath$r$}}    \def \mR {\text{\boldmath$R$}}
\def \svec {\text{\boldmath$s$}}    \def \mS {\text{\boldmath$S$}}
\def \tvec {\text{\boldmath$t$}}    \def \mT {\text{\boldmath$T$}}
\def \uvec {\text{\boldmath$u$}}    \def \mU {\text{\boldmath$U$}}
\def \vvec {\text{\boldmath$v$}}    \def \mV {\text{\boldmath$V$}}
\def \wvec {\text{\boldmath$w$}}    \def \mW {\text{\boldmath$W$}}
\def \xvec {\text{\boldmath$x$}}    \def \mX {\text{\boldmath$X$}}
\def \yvec {\text{\boldmath$y$}}    \def \mY {\text{\boldmath$Y$}}
\def \zvec {\text{\boldmath$z$}}    \def \mZ {\text{\boldmath$Z$}}

 \def \calA {\mathcal A}
 \def \calB {\mathcal B}
 \def \calC {\mathcal C}
 \def \calD {\mathcal D}
 \def \calE {\mathcal E}
 \def \calF {\mathcal F}
 \def \calG {\mathcal G}
 \def \calH {\mathcal H}
 \def \calI {\mathcal I}
 \def \calJ {\mathcal J}
 \def \calK {\mathcal K}
 \def \calL {\mathcal L}
 \def \calM {\mathcal M}
 \def \calN {\mathcal N}
 \def \calO {\mathcal O}
 \def \calP {\mathcal P}
 \def \calQ {\mathcal Q}
 \def \calR {\mathcal R}
 \def \calS {\mathcal S}
 \def \calT {\mathcal T}
 \def \calU {\mathcal U}
 \def \calV {\mathcal V}
 \def \calW {\mathcal W}
 \def \calX {\mathcal X}
 \def \calY {\mathcal Y}
 \def \calZ {\mathcal Z}

\def \ahatvec {\text{\boldmath$\hat a$}}    \def \mhatA {\text{\boldmath$\hat A$}}
\def \bhatvec {\text{\boldmath$\hat b$}}    \def \mhatB {\text{\boldmath$\hat B$}}
\def \chatvec {\text{\boldmath$\hat c$}}    \def \mhatC {\text{\boldmath$\hat C$}}
\def \dhatvec {\text{\boldmath$\hat d$}}    \def \mhatD {\text{\boldmath$\hat D$}}
\def \ehatvec {\text{\boldmath$\hat e$}}    \def \mhatE {\text{\boldmath$\hat E$}}
\def \fhatvec {\text{\boldmath$\hat f$}}    \def \mhatF {\text{\boldmath$\hat F$}}
\def \ghatvec {\text{\boldmath$\hat g$}}    \def \mhatG {\text{\boldmath$\hat G$}}
\def \hhatvec {\text{\boldmath$\hat h$}}    \def \mhatH {\text{\boldmath$\hat H$}}
\def \ihatvec {\text{\boldmath$\hat i$}}    \def \mhatI {\text{\boldmath$\hat I$}}
\def \jhatvec {\text{\boldmath$\hat j$}}    \def \mhatJ {\text{\boldmath$\hat J$}}
\def \khatvec {\text{\boldmath$\hat k$}}    \def \mhatK {\text{\boldmath$\hat K$}}
\def \lhatvec {\text{\boldmath$\hat l$}}    \def \mhatL {\text{\boldmath$\hat L$}}
\def \mhatvec {\text{\boldmath$\hat m$}}    \def \mhatM {\text{\boldmath$\hat M$}}
\def \nhatvec {\text{\boldmath$\hat n$}}    \def \mhatN {\text{\boldmath$\hat N$}}
\def \ohatvec {\text{\boldmath$\hat o$}}    \def \mhatO {\text{\boldmath$\hat O$}}
\def \phatvec {\text{\boldmath$\hat p$}}    \def \mhatP {\text{\boldmath$\hat P$}}
\def \qhatvec {\text{\boldmath$\hat q$}}    \def \mhatQ {\text{\boldmath$\hat Q$}}
\def \rhatvec {\text{\boldmath$\hat r$}}    \def \mhatR {\text{\boldmath$\hat R$}}
\def \shatvec {\text{\boldmath$\hat s$}}    \def \mhatS {\text{\boldmath$\hat S$}}
\def \thatvec {\text{\boldmath$\hat t$}}    \def \mhatT {\text{\boldmath$\hat T$}}
\def \uhatvec {\text{\boldmath$\hat u$}}    \def \mhatU {\text{\boldmath$\hat U$}}
\def \vhatvec {\text{\boldmath$\hat v$}}    \def \mhatV {\text{\boldmath$\hat V$}}
\def \whatvec {\text{\boldmath$\hat w$}}    \def \mhatW {\text{\boldmath$\hat W$}}
\def \xhatvec {\text{\boldmath$\hat x$}}    \def \mhatX {\text{\boldmath$\hat X$}}
\def \yhatvec {\text{\boldmath$\hat y$}}    \def \mhatY {\text{\boldmath$\hat Y$}}
\def \zhatvec {\text{\boldmath$\hat z$}}    \def \mhatZ {\text{\boldmath$\hat Z$}}


\def \atildevec {\text{\boldmath$\tilde a$}}    \def \mtildeA {\text{\boldmath$\tilde A$}}
\def \btildevec {\text{\boldmath$\tilde b$}}    \def \mtildeB {\text{\boldmath$\tilde B$}}
\def \ctildevec {\text{\boldmath$\tilde c$}}    \def \mtildeC {\text{\boldmath$\tilde C$}}
\def \dtildevec {\text{\boldmath$\tilde d$}}    \def \mtildeD {\text{\boldmath$\tilde D$}}
\def \etildevec {\text{\boldmath$\tilde e$}}    \def \mtildeE {\text{\boldmath$\tilde E$}}
\def \ftildevec {\text{\boldmath$\tilde f$}}    \def \mtildeF {\text{\boldmath$\tilde F$}}
\def \gtildevec {\text{\boldmath$\tilde g$}}    \def \mtildeG {\text{\boldmath$\tilde G$}}
\def \htildevec {\text{\boldmath$\tilde h$}}    \def \mtildeH {\text{\boldmath$\tilde H$}}
\def \itildevec {\text{\boldmath$\tilde i$}}    \def \mtildeI {\text{\boldmath$\tilde I$}}
\def \jtildevec {\text{\boldmath$\tilde j$}}    \def \mtildeJ {\text{\boldmath$\tilde J$}}
\def \ktildevec {\text{\boldmath$\tilde k$}}    \def \mtildeK {\text{\boldmath$\tilde K$}}
\def \ltildevec {\text{\boldmath$\tilde l$}}    \def \mtildeL {\text{\boldmath$\tilde L$}}
\def \mtildevec {\text{\boldmath$\tilde m$}}    \def \mtildeM {\text{\boldmath$\tilde M$}}
\def \ntildevec {\text{\boldmath$\tilde n$}}    \def \mtildeN {\text{\boldmath$\tilde N$}}
\def \otildevec {\text{\boldmath$\tilde o$}}    \def \mtildeO {\text{\boldmath$\tilde O$}}
\def \ptildevec {\text{\boldmath$\tilde p$}}    \def \mtildeP {\text{\boldmath$\tilde P$}}
\def \qtildevec {\text{\boldmath$\tilde q$}}    \def \mtildeQ {\text{\boldmath$\tilde Q$}}
\def \rtildevec {\text{\boldmath$\tilde r$}}    \def \mtildeR {\text{\boldmath$\tilde R$}}
\def \stildevec {\text{\boldmath$\tilde s$}}    \def \mtildeS {\text{\boldmath$\tilde S$}}
\def \ttildevec {\text{\boldmath$\tilde t$}}    \def \mtildeT {\text{\boldmath$\tilde T$}}
\def \utildevec {\text{\boldmath$\tilde u$}}    \def \mtildeU {\text{\boldmath$\tilde U$}}
\def \vtildevec {\text{\boldmath$\tilde v$}}    \def \mtildeV {\text{\boldmath$\tilde V$}}
\def \wtildevec {\text{\boldmath$\tilde w$}}    \def \mtildeW {\text{\boldmath$\tilde W$}}
\def \xtildevec {\text{\boldmath$\tilde x$}}    \def \mtildeX {\text{\boldmath$\tilde X$}}
\def \ytildevec {\text{\boldmath$\tilde y$}}    \def \mtildeY {\text{\boldmath$\tilde Y$}}
\def \ztildevec {\text{\boldmath$\tilde z$}}    \def \mtildeZ {\text{\boldmath$\tilde Z$}}

\def \alphavec        {\text{\boldmath$\alpha$}}
\def \betavec         {\text{\boldmath$\beta$}}
\def \gammavec        {\text{\boldmath$\gamma$}}
\def \deltavec        {\text{\boldmath$\delta$}}
\def \epsilonvec      {\text{\boldmath$\epsilon$}}
\def \varepsilonvec   {\text{\boldmath$\varepsilon$}}
\def \zetavec         {\text{\boldmath$\zeta$}}
\def \etavec          {\text{\boldmath$\eta$}}
\def \thetavec        {\text{\boldmath$\theta$}}
\def \varthetavec     {\text{\boldmath$\vartheta$}}
\def \iotavec         {\text{\boldmath$\iota$}}
\def \kappavec        {\text{\boldmath$\kappa$}}
\def \lambdavec       {\text{\boldmath$\lambda$}}
\def \muvec           {\text{\boldmath$\mu$}}
\def \nuvec           {\text{\boldmath$\nu$}}
\def \xivec           {\text{\boldmath$\xi$}}
\def \pivec           {\text{\boldmath$\pi$}}
\def \varpivec        {\text{\boldmath$\varpi$}}
\def \rhovec          {\text{\boldmath$\rho$}}
\def \varrhovec       {\text{\boldmath$\varrho$}}
\def \sigmavec        {\text{\boldmath$\sigma$}}
\def \varsigmavec     {\text{\boldmath$\varsigma$}}
\def \tauvec          {\text{\boldmath$\tau$}}
\def \upsilonvec      {\text{\boldmath$\upsilon$}}
\def \phivec          {\text{\boldmath$\phi$}}
\def \varphivec       {\text{\boldmath$\varphi$}}
\def \psivec          {\text{\boldmath$\psi$}}
\def \chivec          {\text{\boldmath$\chi$}}
\def \omegavec        {\text{\boldmath$\omega$}}

\def \alphahatvec        {\text{\boldmath$\hat \alpha$}}
\def \betahatvec         {\text{\boldmath$\hat \beta$}}
\def \gammahatvec        {\text{\boldmath$\hat \gamma$}}
\def \deltahatvec        {\text{\boldmath$\hat \delta$}}
\def \epsilonhatvec      {\text{\boldmath$\hat \epsilon$}}
\def \varepsilonhatvec   {\text{\boldmath$\hat \varepsilon$}}
\def \zetahatvec         {\text{\boldmath$\hat \zeta$}}
\def \etahatvec          {\text{\boldmath$\hat \eta$}}
\def \thetahatvec        {\text{\boldmath$\hat \theta$}}
\def \varthetahatvec     {\text{\boldmath$\hat \vartheta$}}
\def \iotahatvec         {\text{\boldmath$\hat \iota$}}
\def \kappahatvec        {\text{\boldmath$\hat \kappa$}}
\def \lambdahatvec       {\text{\boldmath$\hat \lambda$}}
\def \muhatvec           {\text{\boldmath$\hat \mu$}}
\def \nuhatvec           {\text{\boldmath$\hat \nu$}}
\def \xihatvec           {\text{\boldmath$\hat \xi$}}
\def \pihatvec           {\text{\boldmath$\hat \pi$}}
\def \varpihatvec        {\text{\boldmath$\hat \varpi$}}
\def \rhohatvec          {\text{\boldmath$\hat \rho$}}
\def \varrhohatvec       {\text{\boldmath$\hat \varrho$}}
\def \sigmahatvec        {\text{\boldmath$\hat \sigma$}}
\def \varsigmahatvec     {\text{\boldmath$\hat \varsigma$}}
\def \tauhatvec          {\text{\boldmath$\hat \tau$}}
\def \upsilonhatvec      {\text{\boldmath$\hat \upsilon$}}
\def \phihatvec          {\text{\boldmath$\hat \phi$}}
\def \varphihatvec       {\text{\boldmath$\hat \varphi$}}
\def \psihatvec          {\text{\boldmath$\hat \psi$}}
\def \chihatvec          {\text{\boldmath$\hat \chi$}}
\def \omegahatvec        {\text{\boldmath$\hat \omega$}}

\def \alphatildevec        {\text{\boldmath$\tilde \alpha$}}
\def \betatildevec         {\text{\boldmath$\tilde \beta$}}
\def \gammatildevec        {\text{\boldmath$\tilde \gamma$}}
\def \deltatildevec        {\text{\boldmath$\tilde \delta$}}
\def \epsilontildevec      {\text{\boldmath$\tilde \epsilon$}}
\def \varepsilontildevec   {\text{\boldmath$\tilde \varepsilon$}}
\def \zetatildevec         {\text{\boldmath$\tilde \zeta$}}
\def \etatildevec          {\text{\boldmath$\tilde \eta$}}
\def \thetatildevec        {\text{\boldmath$\tilde \theta$}}
\def \varthetatildevec     {\text{\boldmath$\tilde \vartheta$}}
\def \iotatildevec         {\text{\boldmath$\tilde \iota$}}
\def \kappatildevec        {\text{\boldmath$\tilde \kappa$}}
\def \lambdatildevec       {\text{\boldmath$\tilde \lambda$}}
\def \mutildevec           {\text{\boldmath$\tilde \mu$}}
\def \nutildevec           {\text{\boldmath$\tilde \nu$}}
\def \xitildevec           {\text{\boldmath$\tilde \xi$}}
\def \pitildevec           {\text{\boldmath$\tilde \pi$}}
\def \varpitildevec        {\text{\boldmath$\tilde \varpi$}}
\def \rhotildevec          {\text{\boldmath$\tilde \rho$}}
\def \varrhotildevec       {\text{\boldmath$\tilde \varrho$}}
\def \sigmatildevec        {\text{\boldmath$\tilde \sigma$}}
\def \varsigmatildevec     {\text{\boldmath$\tilde \varsigma$}}
\def \tautildevec          {\text{\boldmath$\tilde \tau$}}
\def \upsilontildevec      {\text{\boldmath$\tilde \upsilon$}}
\def \phitildevec          {\text{\boldmath$\tilde \phi$}}
\def \varphitildevec       {\text{\boldmath$\tilde \varphi$}}
\def \psitildevec          {\text{\boldmath$\tilde \psi$}}
\def \chitildevec          {\text{\boldmath$\tilde \chi$}}
\def \omegatildevec        {\text{\boldmath$\tilde \omega$}}

\def \mGamma   {\mathbf{\Gamma}}
\def \mDelta   {\mathbf{\Delta}}
\def \mTheta   {\mathbf{\Theta}}
\def \mLambda  {\mathbf{\Lambda}}
\def \mXi      {\mathbf{\Xi}}
\def \mPi      {\mathbf{\Pi}}
\def \mSigma   {\mathbf{\Sigma}}
\def \mUpsilon {\mathbf{\Upsilon}}
\def \mPhi     {\mathbf{\Phi}}
\def \mPsi     {\mathbf{\Psi}}
\def \mOmega   {\mathbf{\Omega}}

\def \mhatGamma   {\mathbf{\hat \Gamma}}
\def \mhatDelta   {\mathbf{\hat \Delta}}
\def \mhatTheta   {\mathbf{\hat \Theta}}
\def \mhatLambda  {\mathbf{\hat \Lambda}}
\def \mhatXi      {\mathbf{\hat \Xi}}
\def \mhatPi      {\mathbf{\hat \Pi}}
\def \mhatSigma   {\mathbf{\hat \Sigma}}
\def \mhatUpsilon {\mathbf{\hat \Upsilon}}
\def \mhatPhi     {\mathbf{\hat \Phi}}
\def \mhatPsi     {\mathbf{\hat \Psi}}
\def \mhatOmega   {\mathbf{\hat \Omega}}

\def \nullvec {\mathbf{0}}
\def \onevec {\mathbf{1}}

%%% theorems
\newtheorem{lem}{Lemma}
\newtheorem{thm}{Theorem}
\newtheorem{coro}{Corollary}
\newtheorem{defn}{Definition}
\newtheorem{remark}{Remark}

\newcommand{\ubar}[1]{\underaccent{\bar}{#1}}



\title{Point-to-point Response \\ JSS 3178}

\author{}

\usepackage{Sweave}
\begin{document}

\maketitle

We are very grateful to the two referees and to the section editor.


\section*{Section Editor}

\noindent
\textit{
My reading of the paper is somewhat different from the reviewers', and
my recommendation is to accept the paper pending minor revisions. That
said, I believe that the reviewers make some useful suggestions for
improving the paper and the author should carefully consider them.}

\textit{More specifically, I think that the difficulty of the paper is largely
intrinsic to the subject-matter, and that the intent of the mlt and
related packages is to implement the methods discussed in as unified,
complete, and general a manner as possible. Nevertheless, reorganizing
the paper along the lines that the reviewers suggest likely would
increase its accessibility and potential impact.}

I thank the section editor and both referees for their helpful suggestions
regarding organisation of the material. The line of arguments in this
manuscript was motivated by negative experiences with earlier, more
theoretical, papers on this subject (especially DOI: 10.1111/sjos.12291).
The classical approach of theory first and examples later discouraged many
readers and so I tried to organise things in reverse order here (not very
successfully, as it seems). However, I'm still convinced that the
example-rich style is a good complement to a theory-driven presention. One
of my aims is to highlight connections between models by showing (almost)
equivalence of empirical results. 

To addess the concerns raised by both referees, I outline the structure of
the manuscript, and the connections between concepts and implementation, in
a flowchart (new Fig 1). In addition, links to existing implementations were
moved from the discussion to the introduction as suggested by both referees.

\textit{Reviewer B also suggests adding user-friendly interfaces for standard
cases, and the author makes reference to this possibility in the last
paragraph of the paper. Whether or not to do this now should be up to
the author, and I can see an argument for separating the basic material
currently in the paper from an implementation meant for a wider
readership.}

\section*{Reviewer A}

\begin{enumerate}

\item \textit{It seems to me the manuscript has two main points.  The first
is that many common statistical models that we usually treat separately can
be viewed in a unified framework.  That framework consists of a family of
transformations of Y and a standard distribution (often N(0,1)) into which
we try to transform Y.}

The manuscript presents, in as much detail as necessary, theoretical results
published recently in a methodological journal (DOI: 10.1111/sjos.12291).
The primary purpose of the manuscript is to introduce and describe the
corresponding software implementation in package mlt.

\item \textit{ The second is an R package for implementing the
framework and estimating its parameters.  I like both of these things.  I
like to be aware of different perspectives on the same ideas and I think it
would be a service to our discipline to publish Hothorn’s perspective.  I
don’t know whether it is new or has been published previously and have not
attempted to find out.  I’m not saying the unified perspective is the best
way to teach statistics, but even if it’s not, it might still be interesting
to people who have previously seen many of the individual models.  I can’t
predict how many statisticians will ultimately use Hothorn’s software, but
its use is not implausible and I suspect at least a few will use it.}

The second purpose of the manuscript is, indeed, to foster an understanding
of many classical models as special cases of a transformation model.

\item \textit{Though I like the ideas, I have concerns about the writing. 
The first has to do with the organization.  Sentence by sentence, the
manuscript is well written.  But I wasn’t able to figure out the main ideas
until at least midway through the ms, and it was the Summary that really
brought them home. I would wish for a better explanation at the start.}

Thank you very much for sharing this experience. It seems that the
reformulation of the normal linear regression model in the introduction was
not sufficient to illustrate the existing connections to other models. My
own re-reading of the manuscript lead to the decision to stay with the
current line of arguments, but to guide readers on their way. I added a
flowchart linking statistical concepts (model formulation, parameter
estimation, interpretation and diagnostics, inference) with sections and
packages/functions/methods discussed.

\item \textit{Second, the Introduction says, “the computationally inaccessible concept
often vanished from the collective consciousness of our profession and the
approximation was taught and understood as the real thing” and the rest of
the Intro reinforces the idea that Hothorn is unhappy with the way many
statisticians think [so am I!] and wants to correct us.  [So do I.]  But
that idea is unnecessary for understanding the rest of the ms and may be
psychologically unwise if the purpose is to get readers to appreciate the
material.  Hothorn may disagree and feel that he wants to address what he
views as sloppy thinking.  But my reading of the ms is that he doesn’t do a
lot to address that problem; the ms is mostly about the unified view and the
software. Readers may be more receptive to the ms if it doesn’t begin by
scolding them. }

\item \textit{I was a little surprised to hear that people might feel
offended by the introduction. In hindsight, I can understand the rather
harsh criticism we received from referees of the corresponding theoretical
paper (the comments are part of https://arxiv.org/pdf/1508.06749.pdf)
in light of your statement better. Maybe my choice of words in the
introduction was a subconscious reaction to the refusal to rethink some of
the traditions we have in statistics. I have, on the other hand, experienced
enthusiastic feedback from younger people who, as you did, value simplicity
and uniformity over mathematical interestingness. I thought about your
comments for quite some time, but I think we won't chance anything by just
being nice. The manscript offers constructive criticism of the way many
models are taught, parameterised, estimated and implemented. The
introduction, with its new paragraph on workflow and organisation as well as
better connections to classical models, will hopefully get readers
hooked.}

\item \textit{Finally, I think the ms could be considerably shortened,
with much of the code either online or in an appendix.  The space saved by
removing code could then be used for more explanation.  }

I agree that a huge proportion of the code serves for presenting existing
implementations and comparison of corresponding results. My intention was to
keep readers in their comfort zone (by analysing data using one of the
classical packages) while showing them the green grass on the other
(transformation) side. I hope that readers will, in the very end, realise:
"Oh, instead of learning of to use all these different packages, it is
enough for me to understand one package!". I admit that only patient readers
willing to carry on until the very end will experience this effect. I tried
to move some of the "light" shining in the discussion to the introduction,
so maybe more people will invest the energy to carry through.

\end{enumerate}

\section*{Reviewer B}

\begin{enumerate}

\item \textit{First I would like to compliment the author on an interesting
manuscript, and interesting package(s).  With that said there are major
issues with the way the manuscript is written from a flow and accessibility
perspective.  }


\item \textit{I enjoy the opening few paragraphs but I believe that the
motivation for the work needs further explained.  Why is this package the
solution to my problems, I believe much of the discussion found in the
summary is clear motivation for this work, and the comparisons that take
shape within Section 2, and should be explained prior to this in the
introduction, as clear motivation.  }

Thank you very much for the suggestion. The introduction was smoothed (a
better connected to the first paragraph was added) and the general workflow
was linked to packages/functions/methods and sections in the manuscript.
Also, some parts of the discussion are now presented in the introduction,
with the aim to link transformation models to models most readers will be
familiar with.

\item \textit{The author jumps into an example from the mlt package with
only really describing the workflow in a single paragraph.  As this paper is
presenting the package more time and effort should be devoted to explaining
why this work flow is important what it allows for.  Your work flow is
complex,  and  is important to the packages.  I believe this would truly
benefit by you taking the time to be in-depth on the work flow before you do
an example.  Or better yet present the workflow through an example at the
simplest level possible without doing a comparison or referring readers to
further sections.  The quality of this work is there, it just needs to be
presented in a simplified manner before introduce complexity.  }

THe workflow is now presented by a flowchart, alongside with the relevant
links to the implementation and description in the manuscript. In some way,
one can understand the whole document as a description of the workflow,
starting with model specification and estimation over model diagnostics and
interpretation to model inference. My experience with the classical approach 
of presenting the theory first and examples later was very unsatisfactory so
far, because many readers won't make it to the first example (I did
therefore not move sections 3-4 before section 2).

\item \textit{Again it seems like section 3 and 4, should be presented
before section 2, and could be broken into a single example or workflow, to
show the impact of the MLT methods.  This would actually give the author a
way to specify how the package integrates to the methods discussed, without
worrying about comparisons to other tools.}

The comparison of the results is not the main focus. My intention was to
show readers how they can fit a model they are already familiar with (here
presented by means of the corresponding function call) using mlt.

\item \textit{I think keeping the focus on MLT
in the early discussion is the most important thing and then using the
examples as justification and comparison is important.  Also some examples
could be moved to an appendix, not to overwhelm the reader in too many
directions.  }

The examples are not just "illustrations", they serve a specific purpose.
The different models fitted to different data sets are stopovers on our
trajectory through the transformation model space. For some (not all!)
readers, this way of thinking about models will hopefully be easier to
understand than a heavily formula-based theoretical description of things
(which is already available from DOI: 10.1111/sjos.12291).

\item \textit{On page 14, the author also discussed the choice of M.  Many
practitioners will read this document for insight, commenting on how to
choose  M, other than it doesn't effect it to choose M too large.  This
raises the question what is large enough, and what will not adversely effect
computational times.  }

\item \textit{A minor issue is the comparisons are hard to see minor
differences and where the effects can be dramatic.  Possibly using smaller
examples or using norms, or differences to define the changes could make
this   Possibly looking at a visualization of this as well.  }

Thank you; relative changes are not presented along with the absolute
values.  For some problems (Figs 5, 6, 7, 11, 12) models have been compared
graphically.

\item \textit{The comment the author makes regarding the usefulness and
impact in the last section makes the work seem more academic than
practitioner friendly.  With that said most of my comments have been
centered around helping combat that as I do see this as useful work, and if
presented in a coherent manner and with a user friendly package the impact
can be attained.  }

\item \textit{The package itself has some flaws in the design which can
impact this as well.  The work flow as previously mentioned has some
complexity that could be reduced by creating a basic wrapper function to do
the "standard" cases.  For instance possibly doing the basis function
definition inside of the wrapper using standard arguments.  This would
simplify the use of the practitioner, while leaving the flexibility for more
complex examples the author portrays.  Also minimizing the number of steps
needed to create the result.  I also suggest the author add an explanation
of the basis functions used and why, this can be as small as a sentence, but
can be impactful.  }

Basis functions are explicitly described at the beginning of Section 2.

\item \textit{Overall I believe these packages show a great use case of
understanding statistical models, and contains interesting content, but the
impact gets lost in the current format of the manuscript and the package. 
Again I congratulate the author on an interesting manuscript.}

\end{enumerate}

\end{document}
